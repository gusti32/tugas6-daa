\documentclass[12pt]{article}

\usepackage[a4paper, total={6.5in, 9in}]{geometry}
\usepackage[indonesian]{babel}
\usepackage{pdfpages}
\usepackage{mathtools}
\usepackage{braket}
\usepackage{slashbox}
\usepackage{comment}
\usepackage{indentfirst}
\usepackage{amssymb}
\usepackage[linesnumbered, ruled, vlined]{algorithm2e}
\usepackage{listings}
\usepackage{xcolor}
\usepackage{hyperref}

\definecolor{codegreen}{rgb}{0,0.6,0}
\definecolor{codegray}{rgb}{0.5,0.5,0.5}
\definecolor{codepurple}{rgb}{0.58,0,0.82}
\definecolor{backcolour}{rgb}{0.95,0.95,0.92}

\lstdefinestyle{mystyle}{
    backgroundcolor=\color{backcolour},   
    commentstyle=\color{codegreen},
    keywordstyle=\color{magenta},
    numberstyle=\tiny\color{codegray},
    stringstyle=\color{codepurple},
    basicstyle=\ttfamily\footnotesize,
    breakatwhitespace=false,         
    breaklines=true,                 
    captionpos=b,                    
    keepspaces=true,                 
    numbers=left,                    
    numbersep=5pt,                  
    showspaces=false,                
    showstringspaces=false,
    showtabs=false,                  
    tabsize=2
}

\lstset{style=mystyle}

\begin{document}
	\begin{figure}[htp]
		\includepdf[fitpaper=true, trim={0 3cm 0 -5cm}]{format_daa.pdf}
	\end{figure}
	\begin{flushleft}
	\end{flushleft}
	\begin{flushleft}
	\end{flushleft}
	\begin{flushleft}
	\end{flushleft}

	\section*{Keunggulan dan Kelemahan dari Brute Force}
	\noindent Keunggulan brute-force:
	\begin{itemize}
		\item Sangat sederhana dan mudah untuk dibuat, tidak pelu memerlukan pengetahuan khusus
		apapun untuk menggunakan atau membuat algoritma jenis ini.
		\item Dapat digunakan untuk menyelesaikan sebagian besar masalah.
		\item Sering digunakan untuk menyelesaikan beberapa masalah seperti pencarian, pengurutan, pencocokan string, dan lain-lain.
		\item Dapat dijadikan sebagai pembanding terhadap algoritma lain.
	\end{itemize}
	Kelemahan brute-force:
	\begin{itemize}
		\item Umumnya tidak efisien dalam menyelesaikan masalah yang besar.
		\item Tidak cocok untuk memecahkan masalah yang memiliki struktur hierarkis dan melibatkan
		operasi logis.
	\end{itemize}
	
	\section*{Karakteristik Brute Force}
	\begin{itemize}
		\item Mudah diimplementasikan.
		\item Melibatkan perhitungan langsung berdasarkan pernyataan masalah dan definisi dari
		konsep yang terlibat.
		\item Mencari semua kemungkinan solusi yang cocok untuk masalah secara keseluruhan.
		\item Umumnya lambat dan tidak efisien. Tidak cocok untuk menyelesaikan masalah yang besar.
		\item Memiliki kompleksitasi waktu yang besar.
	\end{itemize}

	\section*{Penyelesaian Kasus Brute Force}
		
	Sebuah studio musik membuka layanan sewa studio untuk berlatih musik. Penyewa yang hendak berlatih
	harus melakukan pendaftaran dua hari sebelum menggunakan studio. Studio hanya membuka layanan
	selama 14 jam.
	
	Setiap penyewa harus menulis waktu mulai dan waktu berakhir (semua dalam bilangan
	bulat). Karena permintaan sewa sangat banyak, sementara studio hanya melayani satu penyewa, maka
	pemilik studio harus memilih dan menjadwalkan grup band yang akan menyewa. Sehingga sebanyak
	mungkin grup band yang dapat dilayani. Terlampir tabel 10 penyewa.

	\begin{table}[h!]
		\caption{Daftar Penyewa}
		\centering
		\begin{tabular}{ |c|c|c|c|c|c|c|c|c|c|c| }
			\hline
			\textbf{No. Grup Band} & \textbf{1} & \textbf{2} & \textbf{3} & \textbf{4} & \textbf{5} & \textbf{6} & \textbf{7} & \textbf{8} & \textbf{9} & \textbf{10} \\
			\hline
			Mulai    & 1 & 3 & 2 & 4 & 8 & 7 & 9 & 11 & 9 & 12 \\
			\hline				
			Akhir	 & 3 & 4 & 5 & 7 & 9 & 10 & 11 & 12 & 13 & 14 \\
			\hline
		\end{tabular}
	\end{table}	
	
	Jika persoalan diselesaikan dengan algoritma Brute Force, bagaimana caranya dan berapa kompleksitas 
	algoritmanya dalam notasi Big O?
	\newpage
	\subsection*{Jawaban}
	
	\paragraph{Ketentuan} \noindent Studio hanya dapat melayani satu grup band (penyewa) dalam
	rentang waktu sewanya dengan batas waktu pelayanan, yang berarti grup band tidak dapat memakai
	studio jika rentang waktunya bertabrakan dengan rentang waktu grup band lain. Jika
	bertabrakan atau melebihi batas waktu pelayanan maka sesi grup band tersebut akan dipindahkan
	ke hari berikutnya. Sebaliknya, grup band akan diizinkan berlatih pada hari	tersebut jika rentang
	waktunya tidak bertabrakan atau melebihi batas waktu pelayanan.
	
	\noindent \\Berikut ini bentuk dari algoritma brute-force berdasarkan ketentuan di atas:
	
	\begin{algorithm}
		\caption{Algoritma brute-force penjadwalan penyewa studio}
		\KwIn{Waktu mulai ($s$), Waktu selesai ($f$), Batas waktu ($m$)}
		\KwOut{Grup band hari pertama ($A$), Grup band hari berikutnya ($B$)}
		\Begin{
			$n \gets \operatorname{len}(s)$ \tcp*[l]{Mengambil panjang input $s$}
			$t \gets 1$ \tcp*[l]{Waktu terakhir pemakaian studio}
			$A \gets \{\varnothing\}$ \tcp*[l]{Jadwal grup band hari pertama}
			$B \gets \{\varnothing\}$ \tcp*[l]{Jadwal grup band hari berikutnya}
			
			\BlankLine
			\If{$n \neq \operatorname{len}(f)$} {
				exit
			}
			\BlankLine
			\tcp{Lakukan proses brute-force}
			\For{$i \gets 1$ \KwTo $n$} {
				\uIf{$s_{i} \geq t \land f_{i} \leq m$} {
					$A \gets A \cup \{i\}$\;
					$t \gets f_{i}$\;
				}
				\Else{
					$B \gets B \cup \{i\}$\;
				}
			}
			\BlankLine
			\Return $A, B$
		}
	\end{algorithm}
	\subsection*{Proses penyelesaian kasus}
	
	\begin{table}[h!]
		\centering
		\begin{tabular}{ |c|c|c|c|c|c|c|c|c|c|c| }
			\hline
			\textbf{No. Grup Band} & \textbf{1} & \textbf{2} & \textbf{3} & \textbf{4} & \textbf{5} & \textbf{6} & \textbf{7} & \textbf{8} & \textbf{9} & \textbf{10} \\
			\hline
			Mulai ($s$)  & 1 & 3 & 2 & 4 & 8 & 7 & 9 & 11 & 9 & 12 \\
			\hline				
			Akhir ($f$) & 3 & 4 & 5 & 7 & 9 & 10 & 11 & 12 & 13 & 14 \\
			\hline
		\end{tabular}
	\end{table}	
	
	\noindent Memulai dari $i = 1$ dengan $t = 1$ dan $m = 14$
	\begin{itemize}
		\item $i = 1, s_{1} = 1, f_{1} = 3, t = 1$.\\Kondisi $s_{1} \geq t \land f_{1} \leq m$ terpenuhi, maka jadwalkan grup band no. 1 di hari pertama.\\Ubah $t \gets f_1$ dan lanjut ke $i = 2$
		\item $i = 2, s_{2} = 3, f_{2} = 4, t = 3$.\\Kondisi $s_{2} \geq t \land f_{2} \leq m$ terpenuhi, maka jadwalkan grup band no. 2 di hari pertama.\\Ubah $t \gets f_2$ dan lanjut ke $i = 3$
		\item $i = 3, s_{3} = 2, f_{3} = 5, t = 4$.\\Kondisi $s_{3} \geq t \land f_{3} \leq m$ tidak terpenuhi, maka jadwalkan grup band no. 3 di hari berikutnya.\\Lanjut ke $i = 4$
		\item $i = 4, s_{4} = 4, f_{4} = 7, t = 4$.\\Kondisi $s_{4} \geq t \land f_{4} \leq m$ terpenuhi, maka jadwalkan grup band no. 4 di hari pertama.\\Ubah $t \gets f_4$ dan lanjut ke $i = 5$
		\item $i = 5, s_{5} = 8, f_{5} = 9, t = 7$.\\Kondisi $s_{5} \geq t \land f_{5} \leq m$ terpenuhi, maka jadwalkan grup band no. 5 di hari pertama.\\Ubah $t \gets f_5$ dan lanjut ke $i = 6$
		\item $i = 6, s_{6} = 7, f_{6} = 10, t = 9$.\\Kondisi $s_{6} \geq t \land f_{6} \leq m$ tidak terpenuhi, maka jadwalkan grup band no. 6 di hari berikutnya.\\Lanjut ke $i = 7$
		\item $i = 7, s_{7} = 9, f_{7} = 11, t = 9$.\\Kondisi $s_{7} \geq t \land f_{7} \leq m$ terpenuhi, maka jadwalkan grup band no. 7 di hari pertama.\\Ubah $t \gets f_7$ dan lanjut ke $i = 8$
		\item $i = 8, s_{8} = 11, f_{8} = 12, t = 11$.\\Kondisi $s_{8} \geq t \land f_{8} \leq m$ terpenuhi, maka jadwalkan grup band no. 8 di hari pertama.\\Ubah $t \gets f_8$ dan lanjut ke $i = 9$
		\item $i = 9, s_{9} = 9, f_{9} = 13, t = 12$.\\Kondisi $s_{9} \geq t \land f_{9} \leq m$ tidak terpenuhi, maka jadwalkan grup band no. 9 di hari berikutnya.\\Lanjut ke $i = 10$
		\item $i = 10, s_{10} = 12, f_{10} = 14, t = 12$.\\Kondisi $s_{10} \geq t \land f_{10} \leq m$ terpenuhi, maka jadwalkan grup band no. 10 di hari pertama.
	\end{itemize}
	
	\noindent Hasil penjadwalan grup band berdasarkan penyelesaian di atas:
	\noindent \\Jadwal grup band hari pertama $\Rightarrow$ 1, 2, 4, 5, 7, 8, 10
	\noindent \\Jadwal grup band hari berikutnya $\Rightarrow$ 3, 6, 9
	
	\subsection*{Kompleksitas Waktu}
	Algoritma di atas memiliki kompleksitas waktu $O(n)$. Pertumbuhan algoritma ini meningkat secara 
	relatif seiring bertambahnya input.
	
	\newpage
	\subsection*{Implementasi pada Bahasa Pemrograman Python}
	\lstinputlisting[language=Python]{tugas_6_src.py}
	Output:
	\lstinputlisting{tugas_6_out.txt}
	
	Source code dokumen ini dan implementasi dapat diambil dari github saya: \url{https://github.com/gusti32/tugas6-daa}
	
\end{document}